\documentclass{article}

\usepackage[utf8x]{inputenc}
\usepackage[OT1]{fontenc}
\usepackage[french]{babel}
\usepackage{amsfonts}
\usepackage{amsmath}

\title{IFT2035 Rapport du TP1}
\author{Yuchen Hui, Yuyang Xiong}

\newcommand\kw[1]{\textbf{#1}}
\newcommand \Base [2] {\texttt{#1}$_{#2}$}
\newcommand \Hex [1] {\Base{#1}{16}}

\hyphenation{nom-bre sor-tie}

\begin{document}

\maketitle


\section{Remerciment}
Sans aide de la part du professeur et des démonstrateurs, nous n'arriverions jamais à compléter le travail. Veuillez accepter
notre Remerciment par coeur!
\section{Compréhension générale}
Suite à la distribution de l'énoncé et du code de Cadeux, nous avons commencé 
la lecture et la compréhension plus tôt que possible. Notre expérience:



\noindent

\begin{enumerate}
\item 
Nous avons déterminé nos travaux après la lecture de données : Cela cosiste à
implémenter un interpréteur d'un langage Slip de la famille Lisp, dont la syntax
est très similaire à la dernière. La première partie d'un interpréteur, soit l'analyse
lexicale et Syntaxique, a été implantée par le professeur. La sortie de cette analyse 
est une expression sous forme de la structure de donnée Sexp telle que définie dans le code slip.hs.
La fonction 'readSexp' a synthétisé les deux analyses : Application de la fonction readSexp à
une expression du langage Slip retournera une expression Sexp qui décrit l'arbre syntaxique
de l'expression en entrée. Au fur et à mesure, nous avons réalisé que
la connaissance exacte de la syntax de Slip est la base de nos travaux.

\item L'étape suivante : compréhension de la syntax. En lisant la grammaire BNF-like, dans la 
partie 'Survol', nous avons trouvé une erreur : La teminale 'b' est senseé être 
$$b ::= ((tag\ x_1\ x_2\ ...\ x_n)\ e)$$ 
plutôt que 
$$b ::= ((tag\ x_1\ x_2)\ e).$$ 
Une autre remarque importante a été faite plusieurs jours après, quand nous participions à la
vidéoconférence de Maxim : Le filtrage s'applique uniquement aux structures générées par (cons ..).
\item 
La surprise est survenue lorsque nous analysons le sucre syntaxique des appels de fonctions.
Les appels de fonctions, dans Slip, est organisés différement que dans Haskell :
L'argument sera passé par être placé devant la fonction, soit $$(argument\ fonction).$$
Et le sucre syntaxique lui-même est assez étonnant:
$$(e_0\ e_1\ e_2\ ...\ e_n) \Longleftrightarrow (..((e_0\ e_1)\ e_2)\ ... \ e_n)$$
L'argument $e_0$ est propagé de gauche à droit, de la même manière qu'un produit qui traverse
la pipeline industrielle. C'est de l'ordre inverse que dans haskell!
\item La compréhension de la fonction s2l. Le nom s2l implique Sexp to Lexp. Lexp est
une représentation intermédiaire de l'expression à interpréter. Ce que la fonction s2l fait,
c'est de transformer l'arbre syntaxique en Lexp, qui ne contient pas les sucres syntaxiques. 
Par conséquent, nous devrions aussi éliminer les sucres syntaxiques par s2l.

\item La compréhension de la fonction eval. Eval, est une fonction prennant deux environnement
(Statique, dynamique) et un Lexp en argument, qui retourne une 'Value'. Cette phase sert à évaluer
la représentation intermédiaire Lexp et la réduire en 'Value'. Value est une structures de données qui
permet trois formes de résultat de l'évaluation : un Int, une List et une Fonction. 

\end{enumerate}

\section{Implémentation de la fonction s2l}
L'analyse des expression Sexp est critique pour l'implémentation de s2l. Nous avons
adopté la méthodologie d'analyser la structure en Sexp particulière de chaque
type de Slip syntax puis les traiter séparément. Chaque type de Syntax corresponde à
une branche de notre fonction s2l définie en utilisant le filtrage. Donc le premier pas
sera de parfaire nos connaissances de Sexp. Voici des Obstacles que nous avons rencontré et finalement
surmonté : 
\begin{enumerate}
  \item Appel de fonctions. Cette partie de s2l nous semble la plus simple. Cependant, nous n'avons
  aucune idée en termes des façons systématiques d'extraire informations stockées sous forme de Sexp. Au tout début,
  nous pensions que, puisque Sexp représente l'arbre syntaxique d'une expression, alors on pourrait 
  le parcourir tout comme parcourir une structure de donnée arbre. Malheureusement Nous sommes bloqués 
  dedans pour plusieurs heures (Peut-être à cause du fait que jusqu'à date le Cour IFT2015 n'a pas encore
  couvert les matières sur le parcours des arbres). Alors, nous avons changé d'aspect de voir le problème.
  Comme l'indiqué dans l'énoncé (ainsi que la vidéo du professeur), Sexp n'est rien qu'une liste simplement 
  chainée. À partir de cette opinion, nous avons compris les Sexp. Auparavant ce qu'on a considérés comme
  des sousarbres deviennent les sous-liste. Ensuite, inspirés par la fonction foldl pour les liste de haskell,
  nous avons implémenté la fonction 's2l\_foldl', qui prend en argument une autre fonction auxiliaire 'pipeline'
  . Puisque la fonction 's2l\_foldl' est d'ordre supérieur, il sera trop abstrait de décrire en mot. Je vous
  invite donc à lire le code si vous voulez en savoir plus. 
  %%Maintenant, tout est clair : nous devrions d'abord
  %%extraire les duex premiers élément dans la liste (par exemple, $a\ et\ b\ dans\ (a\ b\ c\ \dots)$),appeler 
  %%b à a, qui génère une Lexp k = (Lpipe a b). Ensuite, chaque fois que l'on extrait l'élément suivant 
  %%(on le dénote par s), on ne fait que prendre k et s et sortir (Lpipe k s). Cette procédure est un peu comme
  %%ce que la fonction foldl fait. Inspirés par foldl, nous avons implémenté la fonction s2l\_foldl qui prend 
  %%la fonction 'pipeline' comme un argument et qui parcourt
  %%une liste et applique 'pipeline' aux chaque élément extrait et au réultat retourné par 'pipeline' la dernière itération
  %%(ici récursivité c'est juste comme une boucle) puis appelle récursivement elle même avec résultat retourné par 'pipeline'. 
  %%Par exemple, prend s (élément extrait nouvellement) et k (résultat retourné)
  %%et retourne (Lpipe k s). Puisque foldl est une fonction d'ordre supérieur, il est trop abstrait de décrire par mot, 
  %%nous vous invitons donc à lire le code
  %pour en savoir plus.
  
  \item Constructeur. Avec la fonction 's2l\_foldl', on a 
  qu'à implémenter une fonction auxiliaire 'consConcate' similaire à 'pipeline'
  et appeler 's2l\_foldl' passant en argument 'consConcate'.
  
  
  \item Case. Case utilise aussi le concept de foldl, cependant
  elle est plus complexe. Nous avons modifié 's2l\_foldl' pour
  obtenir la nouvelle fonction 's2l\_foldl\_case'. De la même manière,
  nous avons implémenté des fonctiosn auxiliaires pour case. 

  
  \item Dlet,slet,lambda. Par l'analogie, nous avons trouvé que ces trois branches utilisent le concept de foldr.
  Lambda est la plus simple, un peu comme le rôle que joue l'appel
  des fonctions pour le concept de foldl, donc nous avons implémenté
  's2l\_foldr' pour expression lambda et puis la version adaptée aux slet
  et dlet : 's2l\_foldr\_let'. Aussi, nous avous créé queleques fonctions auxilières.

  
  \item Nous avons remarqué
que la syntax d'un appel de fonction et celle d'une déclaration de varible partage la même forme,
toutefois cela ne causera pas de problème, car les déclarations exsitent seulement dans la branche
pour $(case\ e\ b_1\ ... \ b_n)$ et tout autre ocurrences seront conséquemment considérées comme
un appel de fonction.

\end{enumerate}

\section{Implémentation de la fonction eval}
Avec les expériences des devoirs 1-3, nous sommes tous familiers avec
ce que signifie eval. La partie la plus difficile sera la compréhension des comprtements de
slet et dlet, ainsi que l'évaluation des 'Lfn'.
\subsection{Problème avec le concept de la portée}
Normalement, le concept 'portée' sera utilisé pour décrire une propriété d'un 
langage de programmation. La définition des deux portée dans le livre 'Programming Language Design
Concepts' :

   "A language is statically scoped if the body of a procedure is executed in the environnement of the procedure's definition."

   "A language is dynamically scoped if the body of a procedure is executed in the environnement of the procedure call."

 Cependant, dans l'énoncé du TP, On dit 
'une variale utilise la portée statiques(dynamique).' Cette expression nous
semble plutôt mélangante : On ne parle plus d'évaluation des fonctions mais d'évaluation des variables!
Heureusement, quand nous posons nos questions, le démonstrateur Maxim nous a tracé la lighe
et nous a guidé vers une compréhension correcte : \\
\\
les varibale dans une fonction, si elle est déclarée 
statiquement, nous le cherchons alors dans l'environnement où la fonction a été crée; si elle
est déclarée dynamiquement, Nous le cherchons dans l'environnement où la fonction
est appelée!\\
\\
Après deux jours, nous avons comfirmé notre stipulation, en voyant une réponse rédigée par 
Monsieur Monnier dans le forum de discussion qui nous guide aussi vers cette compréhension! 

\subsection{Choix à faire}
    Lorsque l'on était en train de coder la branche de la fonction eval qui évalue le "Lvar Var", on était bloqué par un problème -- 
    puisqu'on avait deux environnements(l'un est statique et l'autre dynamique), dans quel environnement 
    devait-on chercher le "Var"? On a posé cette question à l'auxiliaire Maxim Bernard pendant la vidéoconférence 
    ayant lieu ce lundi, il croyait qu'on pourrait chercher d'abord dans l'environnement statique puis dans
    l'environnement dynamique. On s'est rappeleé que le prof avait dit, dans la vidéo promenade, qu'il existait 
    plusieurs possibles et que c'était à nous de prendre la décision pour choisir l'environnement, dans lequel 
    on cherche une variable. Finalement, On a décidé de respecter la suggestion de Maxim -- chercher d'abord dans 
    l'environnement statique puis l'environnement dynamique. Nous allons expliqué la raison par
    un exemple ici : \\
    eval env0 [] (s2l (readSexp"(dlet (((curried+ x1) (slet ((x 9)) (lambda (y) (y (x +)))))(x 7))(6 (5 curried+)))"))
    ;$\leadsto$ 15.\\
    Dans cette expression, on remarque que le corps de la fonction curried+ sera (slet ((x 9)) (lambda (y) (y (x +)))),
    où x est déclarée statiquement. En conséquence, quand on veut évaluer le corps, la valeur
    de x devrait être cherchée dans l'environnement où curried+ est définie, dans laquelle x vaut 9. Donc le résultat d'évaluation
    est supposé d'être 15 selon notre compréhension de slet et dlet discutée dans la section 4.1.
    Si on décide de chercher d'abord dans l'environnement statique, l'interpréteur trouve bien que
    x = 9, qui est compatible avec notre compréhension; Par contre, si on cherche premièrement dans l'environnement dynamique,
    l'interpréteur assignera à variable x la valeur 7 trouvée dans l'environnement d'appel. Ce résultat n'est pas
    raisonnable car il est en conflit avec le fait que x est déclarée statiquement. \\
    En conclusion: il vaut mieux de chercher d'abord dans l'environnement statique.

\subsection{Évaluation de Lfn}
 Sur ce qui concerne l'évaluation de Lfn (Par exemple (Lfn "x" (Evar "x"))), Nous ne savions aussi pas trop comment transférer 'x' en Value, 
 vu qu'il n'y a pas de place pour des string dans la structure de donnée Value. Encore une fois, Maxim nous a donné un indice. 
 Avec l'indice, on implémente rapidement le reste du TP1.

\subsection{Autre}
Les autres parties sont similaires aux devoirs 1-3, donc nous n'avons pas rencontré
d'empêchements.
\section{Conclusion}
Grace à ce TP, nous comprenons mieux les concepts des langages de programmaiton
vus en cours en écrivant une partie d'un interpéteur de Slip. Cela nous a pris 5 jours (et un jour pour le rapport et les tests). 
Ce sont 5 jours occupés mais aussi 5 jours remplis de surprise. Merci à tout le monde!



    

    %% \subsection{Allocation des blocs}

%% \begin{enumerate}
%% \item Expliquer comment fonctionne chacune des 4 méthodes d'allocation
%%   suivantes: blocs contigus, blocs chaînés, blocs indexés, \emph{extents}.
%% \item Contraster les avantages et inconvénients de chacune de ces méthodes.
%% \end{enumerate}

%% \subsection{Systèmes de fichiers}

%% Soit un système de fichiers de style ``Unix'' tel que le FFS de BSD ou ext2
%% de Linux, avec des blocs indexés.

%% \begin{enumerate}
%% \item Lister tous les blocs qu'il faut modifier sur le disque lors de
%%   l'exécution de la fonction \texttt{open} qui crée un fichier (de taille
%%   zéro).
%% \item Spécifier l'ordre dans lequel ces opérations devraient être exécutées
%%   pour minimiser l'impact potentiel d'un crash à mi-course.
%% \item Sur la base de l'ordre précédent, indiquer après chaque opération
%%   quels problèmes apparaîtraient en cas de crash à ce moment.
%% %% \item Si le système de fichier est modifié pour garder un \emph{journal},
%% %%   indiquer quelles données devraient être écrites dans le journal pour
%% %%   l'opération précédente.
%% %% \item Estimer le coût en performance d'un tel journal pour cette opération;
%% %%   i.e. indiquer de combien de pourcents l'opération serait-elle ralentie.
%% \end{enumerate}



%% \subsection{Mount multiples (11.2)}

%% \noindent
%% Quels problèmes peuvent apparaître quand on autorise un système de fichiers
%% à être \emph{monté} à plusieurs endroits en même temps.

%% \subsection{VFS (11.8)}

%% \noindent
%% Discuter de l'usage d'une abstraction nommée \emph{VFS} pour permettre à un
%% système d'exploitation d'utiliser facilement plusieurs sortes de systèmes de
%% fichiers.

%% \subsection{Blocs libres (11.11)}

%% \noindent
%% Soit un système où l'espace libre est maintenu dans une liste chaînée de
%% blocs libres.
%% \begin{enumerate}
%% \item Supposons que le pointeur sur le premier bloc libre est perdu.
%%   Le système peut-il reconstruire la liste des blocs libres?
%% %% \item Suggérer un système qui s'assure le pointer sur le premier bloc libre
%% %%   ne peut jamais être perdu à cause d'
%% \end{enumerate}

%% \subsection{Optimisation et pannes (11.13)}

%% \noindent
%% Discuter comment les optimisations de performance pour les systèmes de
%% fichiers peuvent introduire des problèmes de cohérence en cas de panne.

%% \subsection{Indexage indirect progressif (11.15)}

%% \noindent
%% Soit un système de fichiers de type ``Unix File System'' avec des blocs de
%% 8KB, où un pointeur sur un bloc occupe 4 bytes, et où chaque inode contient
%% 12 pointeurs sur des blocs directs, 1 pointeur sur un bloc indirect,
%% 1 pointeur sur un bloc doublement indirect, et 1 pointeur sur un bloc
%% triplement indirect.

%% \begin{itemize}
%% \item Quelle est la taille maximum d'un fichier?
%% \item Combien d'accès disques sont nécessaires (en présumant que le cache
%%   est vide) pour accéder au contenu d'un petit fichier dans \texttt{/a/b/c}?
%% \end{itemize}


%% \subsection*{11.17}

%% \noindent
%% Comparer l'usage d'un \emph{RAM disque} par rapport à l'usage de la même
%% mémoire comme un cache.

%% \subsection*{16.4}

%% \noindent
%% Discuter les avantages et inconvénients de cacher localement les traductions
%% de noms pour les ordinateurs dans les domaines distants.

%% \subsection*{16.16}

%% Discuter les avantages et inconvénients d'utiliser pour les routeurs d'un
%% réseau du matériel dédié plutôt que des ordinateurs standards.

%% \subsection*{16.17}

%% Comparer l'usage de tables statiques pour les noms d'hôtes
%% (\texttt{/etc/hosts}) et l'usage de serveurs de noms.  Discuter comment
%% chacun de ces deux systèmes peut-être ajusté ou augmenté pour bénéficier (au
%% moins en partie) des avantages de l'autre.

%% \subsection*{16.18}

%% Les serveurs de noms sont organisé de manière hiérarchique, pourquoi?

%% \subsection*{16.19}

%% Soit un réseau physique similaire à Ethernet, mais qui retransmet les
%% paquets immédiatement après la détection d'une collision.  Quels problèmes
%% pourraient surgir avec une telle stratégie?  Comment peut-on les rectifier?

%% \subsection*{16.23}

%% Le protocole HTTP original était construit au-dessus de TCP/IP.
%% Pour obtenir une page, le client établi une connection avec le serveur, lui
%% envoie le nom de la page désirée, puis le serveur répond en envoyant la page
%% demandée et en fermant la connection.

%% \begin{itemize}
%% \item Quels problèmes de performance cela pose-t-il?
%% \item Est-ce que UDP serait une bonne alternative?  Pourquoi?
%% \item Quels autres changements peut-on apporter pour améliorer la
%%   performance de HTTP?
%% \end{itemize}

%% \subsection*{16.x}

%% \noindent
%% Sachant qu'un paquet TCP sur un réseau Ethernet peut transporter un maximum
%% de 1460 bytes de données:
%% \begin{itemize}
%% \item Combien de paquets faut-il (dans le meilleur des cas) pour télécharger
%%   un fichier de 10KB d'un serveur web?
%% \end{itemize}

%% \subsection*{16.y}

%% \noindent
%% Combien de paquets faut-il au minimum lorsqu'une machine dont les caches sont
%% vides veut obtenir une page d'un site web tel que \texttt{www.gnu.org}?

%% \subsection*{16.z}

%% \noindent
%% Soit une implantation de TCP qui gère sa fenètre comme suit: la fenêtre
%% initiale est de 2 paquets et elle augmente de 1 pour chaque paquet dont la
%% réception est confirmée, jusqu'à qu'un paquet soit perdu.  Elle divise alors
%% sa fenètre par 2, puis recommence à l'agrandir mais en l'agrandissant non
%% pas de 1 paquet mais de $1 / \textsl{fenêtre}$ paquet pour chaque paquet
%% dont la réception est confirmée, jusqu'à ce qu'un paquet soit perdu.

%% Soit une connection avec un RTT (aller-retour) de $200ms$, une bande
%% passante de $10Mb/s$, des paquets de $1460B$, et en présumant que les
%% routeurs font leur travail de manière parfaite (pas de buffering superflu).

%% \begin{itemize}
%% \item Quelle est la taille de la fenêtre idéale?
%% \item Combien de temps faut-il
%%   pour arriver au premier paquet perdu qui termine la première phase?
%% \item Combien de temps jusqu'au premier paquet perdu qui termine la
%%   deuxième phase?
%% \item Combien de temps aurait été suffisant pour transférer la même quantité
%%   de données si la fenètre avait pu être choisie de manière parfaite dès le
%%   départ au lieu d'utiliser cet algorithme?
%% \item Que se passe-t-il si un routeur a un buffer de 1MB?
%% \end{itemize}

%% \subsection{Réseau IP}

%% \begin{enumerate}
%% \item Décrire les fonctions que doit appeler un processus qui tourne sur la
%%   machine \texttt{frontal.iro.umontreal.ca} pour se connecter au serveur web
%%   de \texttt{www.iro.umontreal.ca}, tous deux connectés au même réseaux
%%   local \emph{ethernet}.
%% \item Décrire quels paquets de quels protocoles sont envoyés et reçus pour
%%   pouvoir établir la connexion précédente.
%% \item Décrire les différences si ce même processus se connecte
%%   à \texttt{www.gnu.org}.
%% \item Estimer le temps nécessaire pour transférer un petite page web entre
%%   ces deux machines, sur la base d'une latence de 1ms sur le réseau local et
%%   50ms entre les deux machines.
%% \end{enumerate}


\end{document}
